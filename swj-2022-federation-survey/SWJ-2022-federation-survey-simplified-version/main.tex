% additional options: [seceqn,secthm,crcready,onecolumn]
% !TeX spellcheck = en_US
% !TeX TXS-program:bibliography = txs:///bibtex
% !BIB TS-program = bibtex
% !BIB program = bibtex

\documentclass[
  sw             % original iosart2x.cls, as per instructions, 1-column
  % twocolum     % original iosart2x.cls, 2-column, no SWJ identifiers
  % swtwocolumn  % patched iosart2x.cls, 2-column, may be unavailable
  % crcready     % to be enabled for camera-ready version
]
{iosart2x}

\usepackage{flushend}
\usepackage[T1]{fontenc}
%\usepackage{times}
\usepackage[utf8]{inputenc}
\usepackage{dcolumn}
%\usepackage{endnotes}
\usepackage{amsmath}
\usepackage{subfigure}
\usepackage{multirow}
%\usepackage{url} % subsumed by hyperref
\usepackage{paralist}
\usepackage{enumerate}
\usepackage{colortbl}
\usepackage{color}
\usepackage[table]{xcolor}
%\usepackage[dvipsnames,svgnames,x11names]{xcolor}
\usepackage{graphicx}
\usepackage{extarrows}
\usepackage{hyperref}
\usepackage{multicol}
\usepackage{ifthen}
\usepackage{caption}
\usepackage{array}
\usepackage{xspace}
\usepackage{xstring}
\usepackage{ulem}
% emph is not underlined
\normalem
\usepackage{soul}
\usepackage{longtable}
\usepackage{booktabs} % for table toprule line
\usepackage{hhline}
\usepackage{bbding}

\usepackage{makecell} %% added by Zhen

\usepackage{multibib}

% -- FORMATTING OPTIONS --

% Value in iosart2x.cls is too low and we get tables pushed at the end
\renewcommand{\floatpagefraction}{0.7}

% "Title Case Headings" vs "Sentence case headings"
%\newcommand{\SC}[2]{#1} % title case (many SWJ papers use it)
\newcommand{\SC}[2]{#2} % sentence case (as per SWJ instructions)

% Headings for dimensions / sub-dimensions in section 3
% (option 1) use sections & paragraphs, good for SWJ
\newcommand{\EvalFrameworkDimensionHeading}[1]{\subsection{#1}}
\newcommand{\EvalFrameworkSubdimensionHeading}[1]{\myparagraph{#1}}
% (option 2) use paragraphs & subparagraphs, bad for SWJ as look the same
%\newcommand{\EvalFrameworkDimensionHeading}[1]{\myparagraph{#1}}
%\newcommand{\EvalFrameworkSubdimensionHeading}[1]{\mysubparagraph{#1}}

% URL style (serif font more compact, as for Data Intell.)
\urlstyle{rm}

\newtheorem{example}{Example}  %% added by Zhen

% -- URL LAST ACCESS DATE --
\newcommand{\LastAccessedDate}{2022-01-18}


% -- SWJ (iosart2x) SETTINGS FROM TEMPLATE --
\pubyear{0000}
\volume{0}
\firstpage{1}
\lastpage{1}



\newcommand{\urlemail}[1]{\href{mailto:#1}{#1}}
\newcommand{\eg}{\textit{e.g.},\xspace}
\newcommand{\ie}{\textit{i.e.},\xspace}
\newenvironment{myinparaenum}
  {\begin{inparaenum}[\itshape (i)]}
  {\end{inparaenum}}
\newenvironment{myenumerate}
  {\begin{enumerate}}%[(1)]}
  {\end{enumerate}}
\newenvironment{myitemize}
  {\begin{itemize}}
  {\end{itemize}}

\definecolor{mygray1}{gray}{.8}
\definecolor{mygray2}{gray}{.9}
\definecolor{mygray3}{gray}{.95}
\definecolor{mygray}{gray}{.9}
\definecolor{mypink}{rgb}{.99,.91,.95}
\definecolor{mycyan}{cmyk}{.3,0,0,0} 

\newcommand{\cmark}{\Checkmark}

\newcommand{\myparagraph}[1]{\paragraph{#1.}}
\newcommand{\mysubparagraph}[1]{\vspace{0.46em plus 0.1em minus 0.1em}\noindent\textit{#1}\hspace{0.46em plus 0.1em minus 0.1em}}

% definitions of data federation systems
\newbool{dfsIsAcademicVal}
\newbool{dfsHasAuthenticationVal}
\newbool{dfsHasAuthorizationVal}
\newbool{dfsHasAuditingVal}
\newbool{dfsHasEncryptionVal}
\newbool{dfsHasDataMaskingVal}
\newbool{dfsHasDataMaskingNGVal}
\newbool{dfsHasGuiVal}
\newbool{dfsHasCliVal}
\newbool{dfsHasJdbcVal}
\newbool{dfsHasOdbcVal}
\newbool{dfsHasWebApiVal}
\newbool{dfsHasAdoNetVal}
\newbool{dfsHasSparqlEndpointVal}
\newbool{dfsIsSupportedVal}
\newbool{dfsIsOpenSourceVal}
\newcommand{\dfsName}[1]{\gdef\dfsNameVal{#1}\ignorespaces}
\newcommand{\dfsRefs}[1]{\gdef\dfsRefsVal{#1}\ignorespaces}
\newcommand{\dfsProvider}[1]{\gdef\dfsProviderVal{#1}\ignorespaces}
\newcommand{\dfsDescription}[1]{\gdef\dfsDescriptionVal{#1}\ignorespaces}
\newcommand{\dfsIsAcademic}{\global\booltrue{dfsIsAcademicVal}\ignorespaces}
\newcommand{\dfsHasAuthentication}{\global\booltrue{dfsHasAuthenticationVal}\ignorespaces}
\newcommand{\dfsHasAuthorization}{\global\booltrue{dfsHasAuthorizationVal}\ignorespaces}
\newcommand{\dfsHasAuditing}{\global\booltrue{dfsHasAuditingVal}\ignorespaces}
\newcommand{\dfsHasEncryption}{\global\booltrue{dfsHasEncryptionVal}\ignorespaces}
\newcommand{\dfsHasDataMasking}{\global\booltrue{dfsHasDataMaskingVal}\ignorespaces}
\newcommand{\dfsHasDataMaskingNG}{\global\booltrue{dfsHasDataMaskingNGVal}\ignorespaces}
\newcommand{\dfsHasGui}{\global\booltrue{dfsHasGuiVal}\ignorespaces}
\newcommand{\dfsHasCli}{\global\booltrue{dfsHasCliVal}\ignorespaces}
\newcommand{\dfsHasJdbc}{\global\booltrue{dfsHasJdbcVal}\ignorespaces}
\newcommand{\dfsHasOdbc}{\global\booltrue{dfsHasOdbcVal}\ignorespaces}
\newcommand{\dfsHasWebApi}{\global\booltrue{dfsHasWebApiVal}\ignorespaces}
\newcommand{\dfsHasAdoNet}{\global\booltrue{dfsHasAdoNetVal}\ignorespaces}
\newcommand{\dfsHasSparqlEndpoint}{\global\booltrue{dfsHasSparqlEndpointVal}\ignorespaces}
\newcommand{\dfsDevelLangs}[1]{\gdef\dfsDevelLangsVal{#1}\ignorespaces}
\newcommand{\dfsIsSupported}{\global\booltrue{dfsIsSupportedVal}\ignorespaces}
\newcommand{\dfsIsOpenSource}{\global\booltrue{dfsIsOpenSourceVal}\ignorespaces}
\newcommand{\dfsDeployOnPremises}[1]{\gdef\dfsDeployOnPremisesVal{#1}\ignorespaces}
\newcommand{\dfsDeployIaasPaas}[1]{\gdef\dfsDeployIaasPaasVal{#1}\ignorespaces}
\newcommand{\dfsDeploySaas}[1]{\gdef\dfsDeploySaasVal{#1}\ignorespaces}
\newcommand{\dfsReleaseFirst}[2]{\gdef\dfsVersionFirstVal{#1}\gdef\dfsYearFirstVal{#2}\ignorespaces}
\newcommand{\dfsReleaseLast}[2]{\gdef\dfsVersionLastVal{#1}\gdef\dfsYearLastVal{#2}\ignorespaces}
\newcommand{\dfsReleaseUnique}[2]{\dfsReleaseFirst{#1}{#2}\dfsReleaseLast{#1}{#2}\ignorespaces}
\newcommand{\dfsQueryLangs}[1]{\gdef\dfsQueryLangsVal{#1}\ignorespaces}
\newcommand{\dfsTransparent}[1]{\gdef\dfsTransparentVal{#1}\ignorespaces} 
\newcommand{\dfsMajorInfrastructural}[1]{\gdef\dfsMajorInfrastructuralVal{#1}\ignorespaces} 
\newcommand{\dfsSrcList}[1]{\gdef\dfsSrcListVal{#1}\ignorespaces}
\newcommand{\dfsPrint}{}
\newcommand{\dfs}[1]{%
	\gdef\dfsNameVal{}%
	\gdef\dfsRefsVal{}%
	\gdef\dfsProviderVal{}%
	\gdef\dfsDescriptionVal{}%
	\global\boolfalse{dfsIsAcademicVal}%
	\global\boolfalse{dfsHasAuthenticationVal}%
	\global\boolfalse{dfsHasAuthorizationVal}%
	\global\boolfalse{dfsHasAuditingVal}%
	\global\boolfalse{dfsHasEncryptionVal}%
	\global\boolfalse{dfsHasDataMaskingVal}%
	\global\boolfalse{dfsHasDataMaskingNGVal}%
	\global\boolfalse{dfsHasGuiVal}%
	\global\boolfalse{dfsHasCliVal}%
	\global\boolfalse{dfsHasJdbcVal}%
	\global\boolfalse{dfsHasOdbcVal}%
	\global\boolfalse{dfsHasWebApiVal}%
	\global\boolfalse{dfsHasAdoNetVal}%
	\global\boolfalse{dfsHasSparqlEndpointVal}%
	\gdef\dfsDevelLangsVal{}%
	\global\boolfalse{dfsIsSupportedVal}%
	\global\boolfalse{dfsIsOpenSourceVal}%
	\gdef\dfsDeployOnPremisesVal{}%
	\gdef\dfsDeployIaasPaasVal{}%
	\gdef\dfsDeploySaasVal{}%
	\gdef\dfsVersionFirstVal{}%
	\gdef\dfsVersionLastVal{}%
	\gdef\dfsYearFirstVal{}%
	\gdef\dfsYearLastVal{}%
	\gdef\dfsQueryLangsVal{}%
	\gdef\dfsSrcListVal{}%
	\ignorespaces #1%
	\dfsPrint%
}

% table column definitions
\newcommand{\PreserveBackslash}[1]{\let\temp=\\#1\let\\=\temp}
\gdef\setRowStyle{} % callback to set row style
\newcolumntype{C}[1]{>{\PreserveBackslash\centering\setRowStyle}m{#1}}
\newcolumntype{R}[1]{>{\PreserveBackslash\raggedleft\setRowStyle}m{#1}}
\newcolumntype{L}[1]{>{\PreserveBackslash\raggedright\setRowStyle}m{#1}}
\newcolumntype{M}{>{\PreserveBackslash\centering\setRowStyle}c} % for multicolumn
\newcolumntype{d}[1]{D{.}{.}{#1}}

% table commands and environments
\gdef\tableColorOdd{gray!10} % was gray!20
\gdef\tableColorEven{gray!25} % was gray!40
\gdef\tableColorTotals{gray!5} % was gray!6
\newcommand{\tabhead}{\gdef\setRowStyle{\baselineskip=1em\bfseries}\setRowStyle}
\newcommand{\tabsubhead}{\gdef\setRowStyle{\fontsize{6.5pt}{6.5pt}\selectfont\baselineskip=1em\bfseries}\setRowStyle}
\newcommand{\tabbody}{\showrowcolors\gdef\setRowStyle{\baselineskip=1em\normalfont}\setRowStyle}
\newcommand{\tabfoot}{\rowcolor{\tableColorTotals}\gdef\setRowStyle{\baselineskip=1em\bfseries}\setRowStyle}
\newenvironment{mytabular}[2][1.2]{
	\scriptsize
	\setlength{\tabcolsep}{3pt}
	\renewcommand{\arraystretch}{#1}
	\rowcolors{1}{\tableColorOdd}{\tableColorEven}
	\begin{tabular}{#2}
		\hiderowcolors
}{
	\end{tabular}%
	\gdef\setRowStyle{}%
	\ignorespaces
}
\newenvironment{mylongtable}[1]{%
	\scriptsize%
	\setlength{\tabcolsep}{3pt}%
	\renewcommand{\arraystretch}{1.2}%
	\rowcolors{1}{\tableColorOdd}{\tableColorEven}
	\begin{longtable}{#1}%
		\hiderowcolors%
}{
	\end{longtable}%
	\gdef\setRowStyle{}%
	\ignorespaces%
}

\makeatletter
\newcommand*\ExpandableInput[1]{\@@input#1 }
\makeatother

\usepackage{mdframed}
\newenvironment{queryexample}{%
	\medskip%
	\begin{mdframed}[
		topline=false,
		bottomline=false,
		rightline=false,
		innertopmargin=5pt,
		innerbottommargin=5pt,
		innerleftmargin=10pt,
		innerrightmargin=5pt,
		linewidth=2pt,
		linecolor=black!20,
		backgroundcolor=black!4
		]\raggedright\small\tt\color{black!70}%
	}{%
	\end{mdframed}%
}

\endinput

%%% Local Variables:
%%% mode: latex
%%% TeX-master: "main"
%%% End:


% ========== DEFINITIONS ==========

% \dfsSrc command
\gdef\dfsSrcTypeFilter{}     % empty, type_letter (filter), ?type_letter ()query)
\gdef\dfsSrcTypeExists{}     % set to non empty when ?type_letter matches
\gdef\dfsSrcPrefixDefault{}  % what to emit before a source, by default
\gdef\dfsSrcPrefixCurrent{}  % what to emit before the source currently being printed
\newcommand{\dfsSrcPrint}[2]{#1\textsubscript{\textit{#2}}}
\newcommand{\dfsSrc}[4][]{%  % [optional_url]{type_letter}{source_name}
  \ifthenelse{\equal{\dfsSrcTypeFilter}{} \OR \equal{\dfsSrcTypeFilter}{#2}}{%
    \dfsSrcPrefixCurrent{}%
    \ifthenelse{\equal{#1}{}}{\dfsSrcPrint{#4}{#3}}{\href{#1}{\dfsSrcPrint{#4}{#3}}}%
    \xdef\dfsSrcPrefixCurrent{\dfsSrcPrefixDefault}%
  }{%
    \ifthenelse{\equal{\dfsSrcTypeFilter}{?#2}}%
      {\xdef\dfsSrcTypeExists{true}}%
      {\ignorespaces}%
  }%
}

% abbreviations
\gdef\srcAbbrMicrosoft{MS}
\gdef\srcAbbrAmazon{Amazon}
\gdef\srcAbbrGoogle{Google}

% source types 
\gdef\srcTypeRelational{Relational\xspace}
\gdef\srcTypeGraph{Graph-based\xspace}
\gdef\srcTypeAggregate{Aggregate-oriented\xspace}
\gdef\srcTypeFile{Structured Files\xspace}
\gdef\srcTypeService{Web Service Paradigms\xspace}
\gdef\srcTypeOther{Other\xspace}

% type_letter:
% r = relational
% g = graph-based
% a = aggregate-oriented
% f = structured files
% w = web service / APIs
% o = other

% flag:
% f = data federation system of this survey
% h = appliance
% m = mdx
% r = RDF triple store
% g = property graph
% k = key/value store
% w = wide column store
% d = document store
% s = search engine
% a = specialized web API


% ========== RELATIONAL ==========

% relational / SQL-based standardized APIs
\gdef\srcADONET{\dfsSrc[https://en.wikipedia.org/wiki/ADO.NET]{r}{}{ADO.NET}}
\gdef\srcJDBC{\dfsSrc[https://en.wikipedia.org/wiki/Java_Database_Connectivity]{r}{}{JDBC}} \gdef\srcODBC{\dfsSrc[https://en.wikipedia.org/wiki/Open_Database_Connectivity]{r}{}{ODBC}}
\gdef\srcOLEDB{\dfsSrc[https://en.wikipedia.org/wiki/OLE_DB]{r}{}{OLE DB}}

% relational / SQL-based RDBMS / open source
\gdef\srcDerby{\dfsSrc[https://en.wikipedia.org/wiki/Apache_derby]{r}{}{Derby}}
\gdef\srcHSQLDB{\dfsSrc[https://en.wikipedia.org/wiki/HSQLDB]{r}{}{HSQLDB}}
\gdef\srcHtwo{\dfsSrc[https://en.wikipedia.org/wiki/H2_(DBMS)]{r}{}{H2}}
\gdef\srcMariaDB{\dfsSrc[https://en.wikipedia.org/wiki/MariaDB]{r}{}{MariaDB}}
\gdef\srcMySQL{\dfsSrc[https://en.wikipedia.org/wiki/MySQL]{r}{}{MySQL}}
\gdef\srcPostgreSQL{\dfsSrc[https://en.wikipedia.org/wiki/PostgreSQL]{r}{}{PostgreSQL}}

% relational / SQL-based RDBMS / proprietary
\gdef\srcActianMatrix{\dfsSrc[http://www.odbms.org/wp-content/uploads/2014/08/WP03-ActianMatrix-0414.pdf]{r}{}{Actian Matrix}}  % formerly ParAccel (PaDB), now discontinued, massively parallel columnar RDBMS for analytic workloads
\gdef\srcActianVector{\dfsSrc[https://en.wikipedia.org/wiki/Vectorwise]{r}{}{Actian Vector}}  % High-performance vectorized columnar analytics database with JDBC driver
\gdef\srcDatacomDB{\dfsSrc[https://en.wikipedia.org/wiki/DATACOM/DB]{r}{}{Datacom/DB}}  % RDBMS for mainframes with JDBC driver (legacy system)
\gdef\srcDBase{\dfsSrc[https://en.wikipedia.org/wiki/DBase]{r}{}{dBASE}}  % old RDBMS introduced for DOS and personal pc
\gdef\srcFirebird{\dfsSrc[https://en.wikipedia.org/wiki/Firebird_(database_server)]{r}{}{Firebird}}
\gdef\srcIBMDBtwo{\dfsSrc[https://en.wikipedia.org/wiki/IBM_Db2_Family]{r}{}{IBM DB2}}
\gdef\srcIBMInformix{\dfsSrc[https://en.wikipedia.org/wiki/IBM_Informix]{r}{}{IBM Informix}}
\gdef\srcIngres{\dfsSrc[https://en.wikipedia.org/wiki/Ingres_(database)]{r}{}{Ingres}}
\gdef\srcMicrosoftAccess{\dfsSrc[https://en.wikipedia.org/wiki/Microsoft_Access]{r}{}{\srcAbbrMicrosoft{} Access}}
\gdef\srcMicrosoftSQLServer{\dfsSrc[https://en.wikipedia.org/wiki/Microsoft_SQL_Server]{r}{$\ast$}{\srcAbbrMicrosoft{} SQL Server}}
\gdef\srcOracleDB{\dfsSrc[https://en.wikipedia.org/wiki/Oracle_Database]{r}{$\ast$}{Oracle DB}}
\gdef\srcOracleTimesTen{\dfsSrc[https://en.wikipedia.org/wiki/TimesTen]{r}{}{Oracle TimesTen}}  % in-memory RDBMS with persistence & high availability with JDBC/ODBC drivers
\gdef\srcParadox{\dfsSrc[https://en.wikipedia.org/wiki/Paradox_(database)]{r}{}{Paradox}}
\gdef\srcProgressOpenEdgeRDBMS{\dfsSrc[https://en.wikipedia.org/wiki/OpenEdge_Advanced_Business_Language]{r}{}{Progress OpenEdge RDBMS}}
\gdef\srcSAPASE{\dfsSrc[https://en.wikipedia.org/wiki/Adaptive_Server_Enterprise]{r}{}{SAP ASE}}
\gdef\srcSAPHANA{\dfsSrc[https://en.wikipedia.org/wiki/SAP_HANA]{r}{$\ast$}{SAP HANA}}
\gdef\srcSAPMaxDB{\dfsSrc[https://en.wikipedia.org/wiki/MaxDB]{r}{}{SAP MaxDB}}  % RDBMS used with SAP applications, also separately available in closed source form
\gdef\srcSASScalablePerformanceDataServer{\dfsSrc[https://documentation.sas.com/doc/en/pgmsascdc/v_022/spdsug/titlepage.htm]{r}{}{SAS Scalable Performance Data Server}}  % SAS server providing parallel access and computation on large datasets, using a relational model and accessible via JDBC
\gdef\srcVertica{\dfsSrc[https://en.wikipedia.org/wiki/Vertica]{r}{}{Vertica}}  % column-store MPP RDBMS with ODBC, JDBC, ADO.NET, and OLE DB interfaces

% relational / SQL-based cluster, cloud & Hadoop stores / open source
\gdef\srcClickHouse{\dfsSrc[https://en.wikipedia.org/wiki/ClickHouse]{r}{}{ClickHouse}}  % OS column-oriented C++ distributed OLAP store (not on Hadoop) powering Yandex, with JDBC driver
\gdef\srcDruid{\dfsSrc[https://en.wikipedia.org/wiki/Apache_Druid]{r}{}{Druid}}  % OS column-oriented Java distributed OLAP store on Hadoop, suitable to time series and event data
\gdef\srcGreenplum{\dfsSrc[https://en.wikipedia.org/wiki/Greenplum]{r}{}{Greenplum}}  % open-source analytic RDBMS built on PostgreSQL (Pivotal Greenplum Database)
\gdef\srcIceberg{\dfsSrc[https://iceberg.apache.org/]{r}{}{Iceberg}} % open-source table format specification, relying on a catalog that may be implemented using HIVE Metadata Store (or a RDBMS) and providing a Java library for manipulating tables; addresses issues of HIVE - not exactly a store but provides the basis of a store
\gdef\srcKudu{\dfsSrc[https://en.wikipedia.org/wiki/Apache_Kudu]{r}{}{Kudu}}  % OS column-oriented Java RDBMS on Hadoop accessible via JDBC using Impala
\gdef\srcPinot{\dfsSrc[https://en.wikipedia.org/wiki/Apache_Pinot]{r}{}{Pinot}}  % OS column-oriented Java distributed OLAP store on Hadoop, with JDBC driver

% relational / SQL-based cluster, cloud & Hadoop stores / proprietary
\gdef\srcAlibabaAnalyticDBforMySQL{\dfsSrc[https://www.alibabacloud.com/product/analyticdb-for-mysql]{r}{}{Alibaba AnalyticDB for MySQL}}  % relational cloud and SQL-based data warehouse solution compatible with MySQL client protocol
\gdef\srcAlibabaDataLakeAnalytics{\dfsSrc[https://www.alibabacloud.com/product/data-lake-analytics]{r}{}{Alibaba Data Lake Analytics}}  % cloud-based analytics service integrated in Alibaba Cloud offer using SQL language and Presto or SPARK as query engines
\gdef\srcAmazonAurora{\dfsSrc[https://en.wikipedia.org/wiki/Amazon_Aurora]{r}{}{\srcAbbrAmazon{} Aurora}}  % cloud RDBMS and document store, compatible with MySQL and PostgreSQL
\gdef\srcAmazonRedshift{\dfsSrc[RDBMS https://en.wikipedia.org/wiki/Amazon_Redshift]{r}{}{\srcAbbrAmazon{} Redshift}}  % large scale data warehouse service for use with business intelligence tools
\gdef\srcDatabricks{\dfsSrc[https://docs.databricks.com/getting-started/introduction/index.html]{r}{}{Databricks}}  % cloud-based data lake platform from the creators of Spark, accessible via JDBC
\gdef\srcExasol{\dfsSrc[https://en.wikipedia.org/wiki/Exasol]{r}{}{Exasol}} % clustered in-memory column-oriented RDBMS
\gdef\srcGoogleBigQuery{\dfsSrc[https://en.wikipedia.org/wiki/BigQuery]{r}{}{\srcAbbrGoogle{} BigQuery}}
\gdef\srcIBMDBtwoWarehouse{\dfsSrc[https://en.wikipedia.org/wiki/IBM_Db2_Family\#Db2_Warehouse]{r}{}{IBM Db2 Warehouse}}  % relational data warehouse available also on cloud
\gdef\srcMicrosoftAzureSQLDatabase{\dfsSrc[https://en.wikipedia.org/wiki/Microsoft_Azure_SQL_Database]{r}{}{\srcAbbrMicrosoft{} Azure SQL Database}}
\gdef\srcMicrosoftSynapseAnalytics{\dfsSrc[https://azure.microsoft.com/en-us/services/synapse-analytics/]{r}{}{\srcAbbrMicrosoft{} Azure Synapse Analytics}}  % a.k.a. SQL Data Warehouse
\gdef\srcSAPIQ{\dfsSrc[https://en.wikipedia.org/wiki/SAP_IQ]{r}{}{SAP IQ}}
\gdef\srcSingleStore{\dfsSrc[https://en.wikipedia.org/wiki/SingleStore]{r}{}{SingleStore}}
\gdef\srcSnowflake{\dfsSrc[https://en.wikipedia.org/wiki/Snowflake_Inc.]{r}{}{Snowflake}}  % cloud-based relational data warehouse with JDBC driver
\gdef\srcTeradata{\dfsSrc[https://www.teradata.com/Vantage]{r}{}{Teradata}}  % hybrid cloud data analytics software platform, primarily relational + document/graph/spatial/time series
\gdef\srcTeradataAster{\dfsSrc[https://assets.teradata.com/resourceCenter/downloads/Brochures/Teradata_Aster_Discovery_Platform_EB7573.pdf]{r}{}{Teradata Aster}}  % discontinued big data for relational data (graph data support claimed too) and different query engines, including one using MapReduce
\gdef\srcTibcoComputeDB{\dfsSrc[https://community.tibco.com/products/tibco-computedb]{r}{}{Tibco ComputeDB}}  % distributed in-memory RDBMS for real-time operational analytics based on Apache Spark, with JDBC driver

% relational / SQL-based query engines or DFS / open source
\gdef\srcDrill{\dfsSrc[https://en.wikipedia.org/wiki/Apache_Drill]{r}{$\ast$}{Drill}}  % schema-free SQL query engine for Hadoop, NoSQL and cloud storage, RDBMS / document-store
\gdef\srcHive{\dfsSrc[https://en.wikipedia.org/wiki/Apache_Hive]{r}{}{Hive}}  % SQL on Hadopp, Java SQL engine suited to complex batch queries, memory+disk processing via MapReduce, Flink or Tez runtimes, with UDF, using Calcite, with JDBC driver
\gdef\srcImpala{\dfsSrc[https://en.wikipedia.org/wiki/Apache_Impala]{r}{}{Impala}}  % SQL on Hadoop, C++/LLVM SQL engine suited for fast interactive queries, in-memory only, no UDF, with JDBC driver
\gdef\srcJBossDataVirtualization{\dfsSrc[https://access.redhat.com/products/red-hat-jboss-data-virtualization]{r}{$\ast$}{JBoss Data Virtualization}}  % a relational DFS of this survey, formerly JBoss Enterprise Data Services Platform
\gdef\srcPresto{\dfsSrc[https://prestodb.io/]{r}{$\ast$}{Presto}}  % Distributed query engine for big data with a relational model and JDBC driver - Facebook / Linux Foundation
\gdef\srcSparkSQL{\dfsSrc[https://en.wikipedia.org/wiki/Apache_Spark\#Spark_SQL]{r}{$\ast$}{Spark}}  % Spark component for managing and querying structured data, via DSL and SQL, with JDBC/ODBC drivers
\gdef\srcTeiid{\dfsSrc[https://teiid.io/]{r}{$\ast$}{Teiid}}  % a system of this survey, presenting a relational interface
\gdef\srcTrino{\dfsSrc[https://trino.io/]{r}{$\ast$}{Trino}}  % Distributed query engine for big data with a relational model and JDBC driver - fork by original authors, commercially exploited by Starburst

% relational / SQL-based query engines or DFS / proprietary
\gdef\srcAmazonAthena{\dfsSrc[https://aws.amazon.com/athena/?nc=sn\&loc=0]{r}{$\ast$}{\srcAbbrAmazon{} Athena}}  % interactive query service (Presto ANSI SQL support) for data on S3, integrated with Amazon Glue and coming with JDBC driver
\gdef\srcDataVirtuality{\dfsSrc[https://datavirtuality.com/en/]{r}{$\ast$}{Data Virtuality}}  % data federation system of this survey, SQL based and likely accessed via its JDBC driver
\gdef\srcDenodo{\dfsSrc[https://www.denodo.com/en/denodo-platform/denodo-platform-80]{r}{$\ast$}{Denodo}} % Denodo Virtual DataPort, a system of this survey
\gdef\srcIBMDbtwoBigSQL{\dfsSrc[https://www.ibm.com/products/db2-big-sql]{r}{$\ast$}{IBM Db2 Big SQL}}  % ANSI-compliant SQL-on-Hadoop engine, included as data federation system in this survey -- DFS SYSTEM
\gdef\srcIBMDVM{\dfsSrc[https://www.ibm.com/products/data-virtualization-manager-for-zos]{r}{}{IBM DVM}}  % IBM Data Virtualization Manager for z/OS - provides virtual, integrated views of data on IBM Z
\gdef\srcMetaMatrix{\dfsSrc[https://en.wikipedia.org/wiki/MetaMatrix]{r}{$\ast$}{MetaMatrix}}  % commercial legacy relational system later acquired by Red Hat and evolved into Teiid
\gdef\srcSASFederationServer{\dfsSrc[https://www.sas.com/en_us/software/federation-server.html]{r}{$\ast$}{SAS Federation Server}}  % a system of this survey, presenting a relational interface
\gdef\srcStarburst{\dfsSrc[https://www.starburst.io/]{r}{$\ast$}{Starburst}}  % DFS of this survey
\gdef\srcTibcoDataVirtualization{\dfsSrc[https://www.tibco.com/products/data-virtualization]{r}{$\ast$}{TibcoDataVirtualization}}  % a system of this survey, presenting a relational interface

% relational / SQL-based appliances
\gdef\srcHPNeoview{\dfsSrc[https://en.wikipedia.org/wiki/HP_Neoview]{r}{h}{HP Neoview}}  % data warehouse HW+SW solution providing RDBMS capabilities and SQL interface, dismissed in 2011
\gdef\srcIBMIntegratedAnalyticsSystem{\dfsSrc[https://www.ibm.com/docs/en/ias]{r}{h}{IBM Integrated Analytics System}}  % HW+SW solution providing RDBMS capabilities, replacement of Netezza
\gdef\srcIBMNetezza{\dfsSrc[https://en.wikipedia.org/wiki/Netezza]{r}{h}{IBM Netezza}}  % data warehouse RDBMS and analytics appliance by IBM with ODBC, JDBC, OLE DB interfaces
\gdef\srcIBMPureData{\dfsSrc[https://en.wikipedia.org/wiki/PureSystems\#PureData]{r}{h}{IBM PureData}}  % HW+SW solutions in different flavours providing OLTP/OLAP RDBMS capabilities
\gdef\srcYellowbrick{\dfsSrc[https://en.wikipedia.org/wiki/Yellowbrick_Data]{r}{h}{Yellowbrick}}  % data warehouse RDBMS with JDBC/ODBC/ADO.NET drivers supplied as cloud server / appliance

% relational / MDX-based sources
\gdef\srcMDX{\dfsSrc[https://en.wikipedia.org/wiki/MultiDimensional_eXpressions]{r}{m}{MDX}}  % OLAP RDBMS accessed via Multidimensional Expressions (MDX) protocol & query language
\gdef\srcMicrosoftAnalysisServices{\dfsSrc[https://docs.microsoft.com/en-us/analysis-services/]{r}{m}{\srcAbbrMicrosoft{} Analysis Service}}  % OLAP server with MDX support, available on Azure or bundled with SQL Server
\gdef\srcMondrian{\dfsSrc[https://en.wikipedia.org/wiki/Mondrian_OLAP_server]{r}{m}{Mondrian}}  % OLAP server with MDX support
\gdef\srcOracleEssbase{\dfsSrc[https://en.wikipedia.org/wiki/Essbase]{r}{m}{Oracle Essbase}}  % multidimensional OLAP RDBMS with JDBC/ODBC/ADO.NET drivers, formerly by Hyperion
\gdef\srcSAPBusinessWarehouse{\dfsSrc[https://en.wikipedia.org/wiki/SAP_NetWeaver_Business_Warehouse]{r}{m}{SAP Business Warehouse}}  % a.k.a. Business Intelligence -- relational data warehouse intended for loading, transforming and storing data modelled into cubes for analysis purposes, also supporting MDX and at a certain point temporarily named SAP BI (Business Intelligence) - see https://www.createch.ca/blog/whats-the-difference-between-sap-bi-sap-bw-sap-bo


% ========== GRAPH-BASED ==========

% graph-based (other systems: JanusGraph, InfiniteGraph, InfoGrid, HypergraphDB, BigData, FlockDB, Graphd, OrientDB also document store)
\gdef\srcAllegroGraph{\dfsSrc[https://en.wikipedia.org/wiki/AllegroGraph]{g}{r$\ast$}{Allegro\-Graph}}
\gdef\srcAmazonNeptune{\dfsSrc[https://en.wikipedia.org/wiki/Amazon_Neptune]{g}{rg$\ast$}{\srcAbbrAmazon{} Neptune}}
\gdef\srcGraphDB{\dfsSrc[https://en.wikipedia.org/wiki/Ontotext_GraphDB]{g}{r}{GraphDB}}
\gdef\srcJenaSource{\dfsSrc[https://jena.apache.org/documentation/javadoc/jena/org/apache/jena/rdf/model/package-summary.html]{g}{}{Jena API}} % Jena Model-compliant
\gdef\srcJenaTDB{\dfsSrc[https://jena.apache.org/documentation/tdb/]{g}{r}{Jena TDB}}
\gdef\srcNeoforj{\dfsSrc[https://en.wikipedia.org/wiki/Neo4j]{g}{g$\ast$}{Neo4j}}
\gdef\srcRDFforJSource{\dfsSrc[https://rdf4j.org/documentation/programming/repository/]{g}{}{RDF4J API}} % RDF4J Repository-compliant
\gdef\srcSparksee{\dfsSrc[https://en.wikipedia.org/wiki/Sparksee_(graph_database)]{g}{g}{Sparksee}}
\gdef\srcSPARQL{\dfsSrc[https://www.w3.org/TR/sparql11-protocol/]{g}{}{SPARQLp}}
\gdef\srcTPF{\dfsSrc[https://linkeddatafragments.org/specification/triple-pattern-fragments/]{g}{}{TPF}}
\gdef\srcStardog{\dfsSrc[https://www.stardog.com/]{g}{r$\ast$}{Stardog}}
\gdef\srcVirtuoso{\dfsSrc[https://en.wikipedia.org/wiki/Virtuoso_Universal_Server]{g}{r$\ast$}{Virtuoso}}


% ========== AGGREGATE-ORIENTED ==========

% aggregate-oriented / key-value stores, a.k.a. distributed caches (other systems: riak, hazelcast, membase, Voldemort, Amazon DynamoDB, Kyoto Cabinet, LevelDB, Memcached, BerkeleyDB, Scalaris, c0treeACE)
\gdef\srcInfinispan{\dfsSrc[https://en.wikipedia.org/wiki/Infinispan]{a}{k}{Infinispan}} % open source distributed cache and key-value NoSQL data store by Red Hat
\gdef\srcOracleNoSQL{\dfsSrc[https://en.wikipedia.org/wiki/Oracle_NoSQL_Database]{a}{k}{Oracle NoSQL}} % key-value store where values can be JSON or table rows, supporting distribution and secondary indexes
\gdef\srcRedHatDataGrid{\dfsSrc[https://developers.redhat.com/products/datagrid/overview]{a}{k}{Red Hat Data Grid}}  % distributed caching solution, commercial version of Infinispan
\gdef\srcRedis{\dfsSrc[https://en.wikipedia.org/wiki/Redis]{a}{k}{Redis}} % distributed key-value store

% aggregate-oriented / wide column stores, a.k.a. column family databases, a.k.a. 2D key-value store (other systems: Hypertable, BigTable)
\gdef\srcAccumulo{\dfsSrc[https://en.wikipedia.org/wiki/Apache_Accumulo]{a}{w}{Accumulo}} % scalable sorted and distributed store based on Google Bigtable
\gdef\srcAmazonSimpleDB{\dfsSrc[https://en.wikipedia.org/wiki/Amazon_SimpleDB]{a}{kw}{\srcAbbrAmazon{} SimpleDB}}  % cloud-based key-value store, still available but being phased out in favor of DynamoDB -- NOTE: sometimes classified as key-value store (but can store/query item columns!) and compared to DynamoDB (which is now a document-store)
\gdef\srcCassandra{\dfsSrc[https://en.wikipedia.org/wiki/Apache_Cassandra]{a}{w}{Cassandra}} % wide-column store based on ideas of BigTable and DynamoDB
\gdef\srcDataStax{\dfsSrc[https://en.wikipedia.org/wiki/DataStax]{a}{w}{DataStax}} % store built on Apache Cassandra
\gdef\srcHBase{\dfsSrc[https://en.wikipedia.org/wiki/Apache_HBase]{a}{w}{HBase}} % wide-column store based on Apache Hadoop and on concepts of BigTable
\gdef\srcMapRDB{\dfsSrc[https://mapr.com/products/mapr-db/]{a}{w}{MapR-DB}} % proprietary wide-column store, apparently derived from HBase and now rebranded

% aggregate-oriented / document stores (other systems: Terrastore, RavenDB, RethinkDB, OrientDB also graph store)
\gdef\srcAmazonDocumentDB{\dfsSrc[https://en.wikipedia.org/wiki/Amazon_DocumentDB]{a}{d}{\srcAbbrAmazon{} DocumentDB}}
\gdef\srcAmazonDynamoDB{\dfsSrc[https://en.wikipedia.org/wiki/Amazon_DynamoDB]{a}{d}{\srcAbbrAmazon{} DynamoDB}}
\gdef\srcCouchbase{\dfsSrc[https://en.wikipedia.org/wiki/Couchbase_Server]{a}{d}{Couchbase}}
\gdef\srcCouchDB{\dfsSrc[https://en.wikipedia.org/wiki/Apache_CouchDB]{a}{d}{CouchDB}}
\gdef\srcMarkLogic{\dfsSrc[https://en.wikipedia.org/wiki/MarkLogic_Server]{a}{d}{MarkLogic}} % also a document store
\gdef\srcMicrosoftAzureCosmosDB{\dfsSrc[https://en.wikipedia.org/wiki/Cosmos_DB]{a}{d}{\srcAbbrMicrosoft{} Azure Cosmos DB}} % multimodel, mainly document
\gdef\srcMongoDB{\dfsSrc[https://en.wikipedia.org/wiki/MongoDB]{a}{d}{MongoDB}}

% aggregate-oriented / search engines
\gdef\srcAmazonOpenSearch{\dfsSrc[https://aws.amazon.com/opensearch-service/]{a}{s}{\srcAbbrAmazon{} OpenSearch}} % Elasticsearch on AWS, OpenSearch = Elasticsearch fork with ASLv2 license
\gdef\srcElasticsearch{\dfsSrc[https://en.wikipedia.org/wiki/Elasticsearch]{a}{s}{Elasticsearch}} % mainly a search engine
\gdef\srcSolr{\dfsSrc[https://en.wikipedia.org/wiki/Apache_Solr]{a}{s}{Solr}} % open source search engine indexing documents consisting of fields
\gdef\srcSplunk{\dfsSrc[https://docs.splunk.com/Documentation/Splunk/]{a}{s}{Splunk}} % indexing and search engine for event/log data, using inverted indexes and storing data as field-based documents


% ========== STRUCTURED FILES ==========

% relational files
\gdef\srcAvro{\dfsSrc[https://en.wikipedia.org/wiki/Apache_Avro]{f}{}{Avro}}  % open source RPC and data serialization framework
\gdef\srcCommonLogFormat{\dfsSrc[https://en.wikipedia.org/wiki/Common_Log_Format]{f}{}{Common Log Format}}  % log file format for Apache & other HTTP servers
\gdef\srcCSV{\dfsSrc[https://en.wikipedia.org/wiki/Comma-separated_values]{f}{}{CSV}}  % CSV/TSV/PSV (P = pipe)
\gdef\srcENVI{\dfsSrc[https://www.l3harrisgeospatial.com/docs/enviimagefiles.html]{f}{}{ENVI}} % binary format for multidimensional raster data
\gdef\srcExcel{\dfsSrc[https://en.wikipedia.org/wiki/Microsoft_Excel]{f}{}{Excel}}  % 
\gdef\srcORC{\dfsSrc[https://en.wikipedia.org/wiki/Apache_ORC]{f}{}{ORC}}  % 
\gdef\srcParquet{\dfsSrc[https://en.wikipedia.org/wiki/Apache_Parquet]{f}{}{Parquet}}  % open source column-oriented data storage format for Hadoop
\gdef\srcRCFile{\dfsSrc[https://en.wikipedia.org/wiki/RCFile]{f}{}{RCFile}}  % file formats for tabular data that splits data by row groups first and then by columns (Record Columnar File)
\gdef\srcSASBDAT{\dfsSrc[https://fileinfo.com/extension/sas7bdat]{f}{}{SAS7BDAT}} % database storage file created by SAS
\gdef\srcSASXPT{\dfsSrc[https://www.loc.gov/preservation/digital/formats/fdd/fdd000464.shtml]{f}{}{SAS XPT}} % SAS Transport File Format Family (XPORT, XPT)
\gdef\srcSequenceFile{\dfsSrc[https://cwiki.apache.org/confluence/display/HADOOP2/SequenceFile]{f}{}{SequenceFile}}  % flat file consisting of binary key/value pairs, used in Hadoop MapReduce

% semi-structured files
\gdef\srcJSON{\dfsSrc[https://en.wikipedia.org/wiki/JSON]{f}{}{JSON}}  %
\gdef\srcXML{\dfsSrc[https://en.wikipedia.org/wiki/XML]{f}{}{XML}}  %
\gdef\srcRDF{\dfsSrc[https://en.wikipedia.org/wiki/Resource_Description_Framework]{f}{}{RDF}}  %

% other files
\gdef\srcAnyFile{\dfsSrc{f}{}{any file (content field + metadata)}}
\gdef\srcMSGEML{\dfsSrc[https://docs.querona.io/data-sources/data-services-files/csv-tsv-pdf-msg-txt.html\#emailmessages]{f}{}{MSG/EML (email)}} % single record with email fields
\gdef\srcPDF{\dfsSrc[https://docs.querona.io/data-sources/data-services-files/csv-tsv-pdf-msg-txt.html\#pdfdocuments]{f}{}{PDF~(metadata)}} % single record with metadata of a PDF file


% ========== WEB APIS / SERVICES ==========

% web / generic web service types
\gdef\srcOData{\dfsSrc[https://en.wikipedia.org/wiki/Open_Data_Protocol]{w}{}{OData}} % HTTP/REST services complying to OData specification
\gdef\srcOpenAPI{\dfsSrc[https://en.wikipedia.org/wiki/OpenAPI_Specification]{w}{}{OpenAPI}} % HTTP/REST services compliant with OpenAPI specification

% web / standardized APIs
\gdef\srcHTTPREST{\dfsSrc[https://en.wikipedia.org/wiki/Representational_state_transfer]{w}{}{HTTP~/ REST}} % arbitrary HTTP-based web services using the REST paradigm, not compliant with OData or OpenAPI
\gdef\srcSOAPWSDL{\dfsSrc[https://en.wikipedia.org/wiki/Web_Services_Description_Language]{w}{}{SOAP~/ WSDL}} % arbitrary XML-based (SOAP), RPC-style web services defined via a WSDL specification


% ========== OTHER ==========

% other / time series DBMS (other systems: InfluxDB)
\gdef\srcAmazonTimestream{\dfsSrc[https://aws.amazon.com/timestream/?nc=sn\&loc=0]{o}{}{\srcAbbrAmazon{} Timestream}} % time series database service for IoT & operational applications with JDBC support
\gdef\srcKdbplus{\dfsSrc[https://en.wikipedia.org/wiki/Kdb\%2B]{o}{}{Kdb+}} % column-based, in-memory relational time series database with SQL like interface and JDBC driver
\gdef\srcOpenTSDB{\dfsSrc[http://opentsdb.net/]{o}{}{OpenTSDB}} % open source time series DBMS based on HBase
\gdef\srcPrometheus{\dfsSrc[https://en.wikipedia.org/wiki/Prometheus_(software)]{o}{}{Prometheus}} % open source time series DBMS & monitoring system for other software components

% other / event stores
\gdef\srcIBMDBtwoEventStore{\dfsSrc[https://en.wikipedia.org/wiki/IBM_Db2_Family\#Db2_Event_Store]{o}{}{IBM Db2 Event Store}} % open source time series DBMS & monitoring system for other software components

% other / multivalue
\gdef\srcSciDB{\dfsSrc[https://en.wikipedia.org/wiki/SciDB]{o}{}{SciDB}} % scalable & distributed DBMS with multi-dimensional array data model

% other / OODBMS, ORM
\gdef\srcInterSystemsCache{\dfsSrc[https://en.wikipedia.org/wiki/InterSystems_Cach\%C3\%A9]{o}{}{InterSystems Cach\'{e}}} % multi-model DBMS, popular as OODMBS and also usable as RDBMS and key-value store
\gdef\srcJPA{\dfsSrc[https://en.wikipedia.org/wiki/Jakarta_Persistence]{o}{}{JPA/JPQL sources}}

% other / legacy navigational DBMS
\gdef\srcBtrieve{\dfsSrc[https://en.wikipedia.org/wiki/Btrieve]{o}{}{Btrieve}}  % navigational DBMS by Pervasive
\gdef\srcIDMS{\dfsSrc[https://en.wikipedia.org/wiki/IDMS]{o}{}{IDMS}}  % network model (CODASYL) DBMS for mainframes (legacy system)
\gdef\srcIMS{\dfsSrc[https://en.wikipedia.org/wiki/IBM_Information_Management_System]{o}{}{IMS}}  % hierarchical DBMS with transaction support

% other / hierarchical content repositories
\gdef\srcModeShape{\dfsSrc[https://modeshape.jboss.org/]{o}{}{ModeShape}} % Open source JCR implementation, supporting a SQL-like language to query nodes in the repository

% other / message-oriented, stream processing
\gdef\srcAmazonKinesis{\dfsSrc[https://aws.amazon.com/kinesis/]{o}{}{\srcAbbrAmazon{} Kinesis}} % cloud service for real-time processing of big data streams
\gdef\srcIBMMQ{\dfsSrc[https://en.wikipedia.org/wiki/IBM_MQ]{o}{}{IBM MQ}}  % message-oriented middleware 
\gdef\srcKafka{\dfsSrc[https://en.wikipedia.org/wiki/Apache_Kafka]{o}{}{Kafka}} % open source stream processing framework
\gdef\srcSAPHANAStreamingAnalytics{\dfsSrc[https://help.sap.com/viewer/5a6dde9fb8be49e180c74f131fe24966/2.0.04/en-US/5cfc455da179443fa85090fcd5823411.html]{o}{}{SAP HANA Streaming Analytics}} % formerly SAP Event Stream Processor https://help.sap.com/viewer/product/SAP_EVENT_STREAM_PROCESSOR/5.1.12/en-US]{SAP HANA Streaming Analytics

% other / directory services
\gdef\srcLDAP{\dfsSrc[https://en.wikipedia.org/wiki/Lightweight_Directory_Access_Protocol]{o}{}{LDAP}}
\gdef\srcMicrosoftActiveDirectory{\dfsSrc[https://en.wikipedia.org/wiki/Active_Directory]{o}{}{\srcAbbrMicrosoft{} Active Directory}}
\gdef\srcRedHatDirectoryServer{\dfsSrc[https://en.wikipedia.org/wiki/389_Directory_Server]{o}{}{Red Hat Directory Server}}

% other / open, non-database standardized protocols
\gdef\srcIMAP{\dfsSrc[https://en.wikipedia.org/wiki/Internet_Message_Access_Protocol]{o}{}{IMAP}}
\gdef\srcRSS{\dfsSrc[https://en.wikipedia.org/wiki/RSS]{o}{}{RSS}}

% other / specialized sources, mostly commercial services accessed via web APIs 
\gdef\srcAmazonAWSSystemsManagerInventory{\dfsSrc[https://docs.aws.amazon.com/systems-manager/latest/userguide/systems-manager-inventory.html]{o}{a}{\srcAbbrAmazon{} AWS System Manager Inventory}} % cloud service collecting metadata of all resources used by an AWS customer
\gdef\srcAmazonCloudWatch{\dfsSrc[https://en.wikipedia.org/wiki/Amazon_Elastic_Compute_Cloud\#Amazon_CloudWatch]{o}{a}{Amazon CloudWatch}} % cloud service collecting logs and metrics from other machines/services
\gdef\srcBioRS{\dfsSrc[https://www.biomax.com/lib/products/biors/Biomax_BioRS_Technical_Profile.pdf]{o}{}{BioRS}}  % query & retrieval system for integrating biological and medical data by Biomax Informatics (dismissed or merged into other product)
\gdef\srcDHLTrackAndTraceAPI{\dfsSrc[https://developer.dhl.com/api-reference/shipment-tracking]{o}{a}{DHL Track \& Trace}}
\gdef\srcEloqua{\dfsSrc[https://www.oracle.com/cx/marketing/automation/]{o}{a}{Eloqua}}  % SaaS platform for marketing automation for supporting marketing campaigns and sales lead generation, from Oracle
\gdef\srcFacebook{\dfsSrc[https://developers.facebook.com/docs/graph-api/]{o}{a}{Facebook}}
\gdef\srcGoogleAdsAPI{\dfsSrc[https://developers.google.com/google-ads/api/]{o}{a}{\srcAbbrGoogle{} Ads}}
\gdef\srcGoogleAnalyticsAPI{\dfsSrc[https://developers.google.com/analytics/]{o}{a}{\srcAbbrGoogle{} Analytics}}
\gdef\srcGoogleCalendarAPI{\dfsSrc[https://developers.google.com/calendar/api]{o}{a}{\srcAbbrGoogle{} Calendar}}
\gdef\srcGoogleContactsAPI{\dfsSrc[https://developers.google.com/contacts]{o}{a}{\srcAbbrGoogle{} Contacts}}  % now migrated to People API https://developers.google.com/people
\gdef\srcGoogleSheetsAPI{\dfsSrc[https://developers.google.com/sheets/api]{o}{a}{\srcAbbrGoogle{} Sheets}}
\gdef\srcHubSpot{\dfsSrc[https://en.wikipedia.org/wiki/HubSpot]{o}{a}{HubSpot}}  % software for CRM, inbound marketing, and sales
\gdef\srcITPilot{\dfsSrc[https://community.denodo.com/docs/html/browse/6.0/itpilot/user_guide/index]{o}{}{ITPilot (website wrapper generator)}}
\gdef\srcJira{\dfsSrc[https://en.wikipedia.org/wiki/Jira_(software)]{o}{a}{Jira}}
\gdef\srcMarketo{\dfsSrc[https://en.wikipedia.org/wiki/Marketo]{o}{a}{Marketo}}  % marketing automation software
\gdef\srcMicrosoftSharepoint{\dfsSrc[https://en.wikipedia.org/wiki/Excel_Services]{o}{a}{\srcAbbrMicrosoft{} Sharepoint}}
\gdef\srcMicrosoftSharepointExcelServices{\dfsSrc[https://en.wikipedia.org/wiki/Excel_Services]{o}{a}{\srcAbbrMicrosoft{} Sharepoint Excel Services}}
\gdef\srcNetsuite{\dfsSrc[https://en.wikipedia.org/wiki/NetSuite]{o}{a}{NetSuite}}  % cloud business management suite comprising ERP/Financials, CRM and ecommerce, used by more than 28,000 customers, owned by Oracle
\gdef\srcOSIsoftPI{\dfsSrc[https://en.wikipedia.org/wiki/OSIsoft]{o}{}{OSIsoft PI}}  % suite of software products for capturing, processing, analyzing, and storing real-time data, focusing in industrial settings and with data exposed (also) via JDBC interface (assuming a fixed schema)
\gdef\srcSalesforce{\dfsSrc[https://www.salesforce.com/products/marketing-cloud/pricing/salesforce-cdp/]{o}{a}{Salesforce}}  % cloud data source with relational model and support for ODBC/JDBC ccess (Salesforce CDP - Customer Data Platform)
\gdef\srcSAPBAPI{\dfsSrc[https://en.wikipedia.org/wiki/Business_Application_Programming_Interface]{o}{a}{SAP Business}}
\gdef\srcSAPGatewayOData{\dfsSrc[https://help.sap.com/doc/05d53b2d3bbb43d2ab5efa23829b2777/1610\%20002/en-US/frameset.htm?ecaeea50ca692309e10000000a445394.html]{o}{a}{SAP Gateway OData}}
\gdef\srcSAPRFC{\dfsSrc[https://support.sap.com/en/product/connectors/nwrfcsdk.html]{o}{a}{SAP RFC}} % Remote Function Call
\gdef\srcTwitter{\dfsSrc[https://en.wikipedia.org/wiki/Twitter]{o}{a}{Twitter}}


\newcites{art}{Articles-Books-Collections}
\newcites{mis}{Theses-Technical-Reports}
\newcites{proc}{Proceedings}

\begin{document}

\begin{frontmatter}

\title{A \SC{S}{s}ystematic \SC{O}{o}verview of \SC{D}{d}ata \SC{F}{f}ederation \SC{S}{s}ystems
--- {\small \textbf{A simplified version}}
}

\runtitle{A \SC{S}{s}ystematic \SC{O}{o}verview of \SC{D}{d}ata \SC{F}{f}ederation \SC{S}{s}ystems}

% {zhgu@unibz.it}
% {xiao@inf.unibz.it}
% {lanti@inf.unibz.it}
% Alessandro.Mosca@unibz.it
% Jing.Xiong@unibz.it
% {calvanese@inf.unibz.it}

\begin{aug}
\author[A]{\inits{Z.}\fnms{Zhenzhen}  \snm{Gu}\ead[label=ezg]{zhenzhen.gu@unibz.it}\ead[label=eunibz]{<name>.<surname>@unibz.it}}
  \author[A]{\inits{F.}\fnms{Francesco} \snm{Corcoglioniti}\ead[label=efc]{francesco.corcoglioniti@unibz.it}}
\author[A]{\inits{D.}\fnms{Davide} \snm{Lanti}\ead[label=edl]{davide.lanti@unibz.it}}
\author[A]{\inits{A.}\fnms{Alessandro} \snm{Mosca}\ead[label=eam]{alessandro.mosca@unibz.it}}
\author[B,C,D]{\inits{G.}\fnms{Guohui} \snm{Xiao}\ead[label=egx]{guohui.xiao@uib.no}\ead[label=euio]{guohuix@ifi.uio.no}\ead[label=eontopic]{<name>.<surname>@ontopic.ai}\thanks{Corresponding author. \printead{egx}.}}
\author[A]{\inits{J.}\fnms{Jing} \snm{Xiong}\ead[label=ejx]{jing.xiong@unibz.it}}
\author[A,D,E]{\inits{D.}\fnms{Diego} \snm{Calvanese}\ead[label=edc]{diego.calvanese@unibz.it}\ead[label=eumea]{diego.calvanese@umu.se}}
%
\address[A]{KRDB Research Centre, Faculty of Computer Science, \orgname{Free University of Bozen-Bolzano}, \cny{Italy}\printead[presep={\\}]{eunibz}}
\address[B]{Department of Information Science and Media Studies, \orgname{University of Bergen}, \cny{Norway}\printead[presep={\\}]{egx}}
\address[C]{Department of Informatics, \orgname{University of Oslo}, \cny{Norway}\printead[presep={\\}]{euio}}
\address[D]{\orgname{Ontopic S.r.l}, \cny{Italy}\printead[presep={\\}]{eontopic}}
\address[E]{Department of Computing Science, \orgname{Ume\aa\ University}, \cny{Sweden}\printead[presep={\\}]{eumea}}
\end{aug}

\end{frontmatter}

\noindent
This document is a simplified version of the paper published in the Semantic Web Journal and provided a systematic overview of 50 data federation systems. The motivation of this document is to provide a way of making users, such as data federation consumers and developers, to check or get knowledge of the characteristics of the considered data federation systems quickly rather than reading a paper with 60 pages. 


\begin{table*}[tbp]
	\centering
	\caption{Summary of the selected data federation systems. Academic systems in \textit{italics}}
	\label{Tab:NewSummaryOfSystem}
	\renewcommand{\dfsPrint}{
		\ifdefempty{\dfsNameVal}{}{\ifbool{dfsIsAcademicVal}{\textit{\dfsNameVal}}{\dfsNameVal}}%
                \ifdefempty{\dfsRefsVal}{}{\,\cite{\dfsRefsVal}} &
		\ifdefempty{\dfsProviderVal}{}{\dfsProviderVal} &
		\ifdefempty{\dfsDescriptionVal}{}{\dfsDescriptionVal} \\
	}
	\begin{mytabular}[1.19]{|L{2.95cm}||L{2.95cm}||L{9.3cm}|}
		\hhline{|-||-||-|}
		\tabhead
		System & Provider & Description \\
		\hhline{:=::=::=:}
		\tabbody
		\ExpandableInput{Tab/systems.tex}
		\hhline{|-||-||-|}
	\end{mytabular}
\end{table*}

\endinput

%%% Local Variables:
%%% mode: latex
%%% TeX-master: "../main"
%%% End:


% !TeX spellcheck = en_US

\begin{table*}[tbp]
        \centering
        \caption{Evaluation of query language and data source
         sub-dimensions. Academic systems in \textit{italics}. ``--''~denotes
         feature/information not found in the systems' official documentation,
         websites, or academic publications, to the best of our efforts}
        \label{Tab:DataFederationDimension}

        \newcounter{dfsQueryLangsSparqlCount}
        \newcounter{dfsQueryLangsSqlCount}
        \newcounter{dfsQueryLangsOtherCount}
        \newcounter{dfsSrcRelationalCount}
        \newcounter{dfsSrcGraphCount}
        \newcounter{dfsSrcAggregateCount}
        \newcounter{dfsSrcFileCount}
        \newcounter{dfsSrcServiceCount}
        \newcounter{dfsSrcOtherCount}
        \newcommand{\dfsSrcTypeCheckmark}[2]{%
            \gdef\dfsSrcTypeFilter{#1}%
            \gdef\dfsSrcTypeExists{}%
            \dfsSrcListVal{}%
            %\ifbool{dfsIsAcademicVal}{ % only academic systems
            \ifdefempty{\dfsSrcTypeExists}{--}{\cmark\addtocounter{#2}{1}}%
            %}{--}
        }
        
        \renewcommand{\dfsPrint}{
                \StrSubstitute{ \dfsQueryLangsVal,@@}{ SPARQL,}{}[\dfsTmp]
                \StrSubstitute{\dfsTmp}{ SQL,}{}[\dfsTmp]
                \StrSubstitute{\dfsTmp}{ BGPs,}{}[\dfsTmp]
                \StrSubstitute{\dfsTmp}{,@@}{}[\dfsTmp]
                \StrSubstitute{\dfsTmp}{@@}{}[\dfsTmp]
                \xdef\dfsQueryLangsOtherVal{\dfsTmp}
                \StrSubstitute{\dfsTmp}{ }{}[\dfsTmp]
                \IfStrEq{\dfsTmp}{}{\xdef\dfsQueryLangsOtherVal{}}{}
                %
                \ifdefempty{\dfsNameVal}{}{\ifbool{dfsIsAcademicVal}{\textit{\dfsNameVal}}{\dfsNameVal}} &
                \IfSubStr{ \dfsQueryLangsVal,}{ SPARQL,}{%
                    \cmark\addtocounter{dfsQueryLangsSparqlCount}{1}%
                }{
                    \IfSubStr{ \dfsQueryLangsVal,}{ BGPs,}{%
                        \cmark\addtocounter{dfsQueryLangsSparqlCount}{1}\hspace{-.05cm}$_\textit{bgp}$\hspace{-.28cm}\,%
                    }{--}
                } &
                \IfSubStr{ \dfsQueryLangsVal,}{ SQL,}{%
                    \cmark\addtocounter{dfsQueryLangsSqlCount}{1}%
                }{--} &
                \ifdefempty{\dfsQueryLangsOtherVal}{--}{\dfsQueryLangsOtherVal\addtocounter{dfsQueryLangsOtherCount}{1}} &
                \dfsSrcTypeCheckmark{?r}{dfsSrcRelationalCount} &
                \dfsSrcTypeCheckmark{?g}{dfsSrcGraphCount} &
                \dfsSrcTypeCheckmark{?a}{dfsSrcAggregateCount} &
                \dfsSrcTypeCheckmark{?f}{dfsSrcFileCount} &
                \dfsSrcTypeCheckmark{?w}{dfsSrcServiceCount} &
                \dfsSrcTypeCheckmark{?o}{dfsSrcOtherCount} \\
        }

        \begin{mytabular}{|L{3.5cm}||C{0.9cm}C{0.9cm}C{1.8cm}||C{1.1cm}C{1.1cm}C{1.1cm}C{1.1cm}C{1.4cm}C{0.9cm}|}
        %\begin{mytabular}{|L{2.8cm}||C{0.8cm}C{0.6cm}C{1.3cm}||C{0.9cm}C{1cm}C{1cm}C{1.1cm}C{1.3cm}C{1cm}C{0.8cm}|}
                \hhline{-||---||------}
                \tabhead
                \multirow{3}{*}{System} & \multicolumn{3}{M||}{Query language} & \multicolumn{6}{M|}{Data source}  \\
                \hhline{|~||---||------|}
                \tabsubhead
                & SPARQL & SQL & Other & \srcTypeRelational & \srcTypeGraph & \srcTypeAggregate & \srcTypeFile & \srcTypeService & \srcTypeOther \\
                \hhline{:=::===::======:}
                \tabbody
                \ExpandableInput{Tab/systems.tex}
                \hhline{:=::===::======:}
                \tabfoot
                Number &
                \arabic{dfsQueryLangsSparqlCount} &
                \arabic{dfsQueryLangsSqlCount} &
                \arabic{dfsQueryLangsOtherCount} &
                \arabic{dfsSrcRelationalCount} &
                \arabic{dfsSrcGraphCount} &
                \arabic{dfsSrcAggregateCount} &
                \arabic{dfsSrcFileCount} &
                \arabic{dfsSrcServiceCount} &
                \arabic{dfsSrcOtherCount} \\
                \hhline{|-||---||------|}
        \end{mytabular}
\end{table*}

\endinput

%%% Local Variables:
%%% mode: latex
%%% TeX-master: "../main"
%%% End:



% !TeX spellcheck = en_US

\begin{table*}[tbp]
        \centering
        \caption{Evaluation of the data security dimension. Academic systems in
         \textit{italics}. ``--''~denotes feature/information not found in the
         systems' official documentation, websites, or academic publications,
         to the best of our efforts.
         Subscript $_{\textit{ng}}$ denotes the use of named graph-based solutions to hide (mask) sensitive information in selected graphs to certain users, and possibly (for AnzoGraph DB) expose sanitized named graph views
        }
        \label{Tab:DataManagementDimension}

        \newcounter{dfsHasAuthenticationCount}
        \newcounter{dfsHasAuthorizationCount}
        \newcounter{dfsHasAuditingCount}
        \newcounter{dfsHasEncryptionCount}
        \newcounter{dfsHasDataMaskingCount}

        \renewcommand{\dfsPrint}{
                \ifdefempty{\dfsNameVal}{}{\ifbool{dfsIsAcademicVal}{\textit{\dfsNameVal}}{\dfsNameVal}} &
                \ifbool{dfsHasAuthenticationVal}{\cmark\addtocounter{dfsHasAuthenticationCount}{1}}{--} &
                \ifbool{dfsHasAuthorizationVal}{\cmark\addtocounter{dfsHasAuthorizationCount}{1}}{--} &
                \ifbool{dfsHasAuditingVal}{\cmark\addtocounter{dfsHasAuditingCount}{1}}{--} &
                \ifbool{dfsHasEncryptionVal}{\cmark\addtocounter{dfsHasEncryptionCount}{1}}{--} &
                \ifbool{dfsHasDataMaskingVal}{\cmark\addtocounter{dfsHasDataMaskingCount}{1}}{--}% 
                \ifbool{dfsHasDataMaskingNGVal}{\hspace{-.05cm}$_\textit{ng}$\hspace{-.2cm}\,}{} 
                \\
        }

        \begin{mytabular}{|L{3.7cm}||C{2.2cm}C{2.2cm}C{2.2cm}C{2.2cm}C{2.2cm}|}
        %\begin{mytabular}{|L{3.5cm}||C{2.0cm}C{2.0cm}C{2.0cm}C{2.0cm}C{2.0cm}|}
                \hhline{|-||-----|}
                \tabhead
                \multirow{2}{*}{System} & \multicolumn{5}{M|}{Data security} \\
                \hhline{|~||-----|}
                \tabsubhead
                & Authentication & Authorization & Auditing & Encryption & Data masking \\
                \hhline{:=::=====:}
                \tabbody
                \ExpandableInput{Tab/systems.tex}
                \hhline{:=::=====:}
                \tabfoot
                Number &
                \arabic{dfsHasAuthenticationCount} &
                \arabic{dfsHasAuthorizationCount} &
                \arabic{dfsHasAuditingCount} &
                \arabic{dfsHasEncryptionCount} &
                \arabic{dfsHasDataMaskingCount} \\
                \hhline{|-||-----|}
        \end{mytabular}
\end{table*}

\endinput

%%% Local Variables:
%%% mode: latex
%%% TeX-master: "../main"
%%% End:



% !TeX spellcheck = en_US

%OLE DB Driver; OData; Node.js; ADO.NET

\begin{table*}[tbp]
        \centering
        \caption{Evaluation of the \textit{interface} dimension. Academic
         systems in \textit{italics}. ``--''~denotes feature/information not
         found in the systems' official documentation, websites, or academic publications, to the best of our efforts}
        \label{Tab:InterfaceDimension}

        \newcounter{dfsHasGuiCount}
        \newcounter{dfsHasCliCount}
        \newcounter{dfsHasJdbcCount}
        \newcounter{dfsHasOdbcCount}
        \newcounter{dfsHasWebApiCount}
        \newcounter{dfsHasAdoNetCount}
        \newcounter{dfsHasSparqlEndpointCount}

        \renewcommand{\dfsPrint}{
                \ifdefempty{\dfsNameVal}{}{\ifbool{dfsIsAcademicVal}{\textit{\dfsNameVal}}{\dfsNameVal}} &
                \ifbool{dfsHasGuiVal}{\cmark\addtocounter{dfsHasGuiCount}{1}}{--} &
                \ifbool{dfsHasCliVal}{\cmark\addtocounter{dfsHasCliCount}{1}}{--} &
                \ifbool{dfsHasJdbcVal}{\cmark\addtocounter{dfsHasJdbcCount}{1}}{--} &
                \ifbool{dfsHasOdbcVal}{\cmark\addtocounter{dfsHasOdbcCount}{1}}{--} &
                \ifbool{dfsHasWebApiVal}{\cmark\addtocounter{dfsHasWebApiCount}{1}}{--} &
                \ifbool{dfsHasAdoNetVal}{\cmark\addtocounter{dfsHasAdoNetCount}{1}}{--} &
                \ifbool{dfsHasSparqlEndpointVal}{\cmark\addtocounter{dfsHasSparqlEndpointCount}{1}}{--} \\
        }

        \begin{mytabular}{|L{3.7cm}||C{1.5cm}C{1.5cm}||C{1.5cm}C{1.5cm}C{1.5cm}C{1.5cm}C{1.5cm}|}
        %\begin{mytabular}{|L{2.8cm}||C{1.5cm}C{1.5cm}||C{1.5cm}C{1.5cm}C{1.5cm}C{1.5cm}C{1.5cm}|}
                \hhline{|-||--||-----|}
                \tabhead
                \multirow{3}{=}{System} & \multirow{3}{=}{\centering Graphical interface} & \multirow{3}{=}{\centering Command line interface} & \multicolumn{5}{M|}{Application programming interface} \\
                \hhline{|~||~~||-----|}
                \tabsubhead
                & & & JDBC Driver & ODBC Driver & Web API & ADO.NET & SPARQL HTTP API \\
                \hhline{:=::==::=====:}
                \tabbody
                \ExpandableInput{Tab/systems.tex}
                \hhline{:=::==::=====:}
                \tabfoot
                Number &
                \arabic{dfsHasGuiCount} &
                \arabic{dfsHasCliCount} &
                \arabic{dfsHasJdbcCount} &
                \arabic{dfsHasOdbcCount} &
                \arabic{dfsHasWebApiCount} &
                \arabic{dfsHasAdoNetCount} &
                \arabic{dfsHasSparqlEndpointCount} \\
                \hhline{|-||--||-----|}
        \end{mytabular}
\end{table*}

\endinput

%%% Local Variables:
%%% mode: latex
%%% TeX-master: "../main"
%%% End:



% !TeX spellcheck = en_US

\begin{table*}[tbp]
        \centering
        \caption{%
            Evaluation of \textit{development} dimension. Academic systems in \textit{italics}.
            ``F." and ``L." denote ``First" and ``Latest" respectively.
            Subscript letters further qualify available deployment options: \textit{n} = native; \textit{c} = containerized; \textit{a} = Amazon AWS; \textit{m} = Microsoft Azure; \textit{g} = Google Cloud Platform. 
            ``--''~denotes feature/information not found in the systems' official documentation, websites, or academic publications, to the best of our efforts%
        }
        \label{Tab:DevelopmentDimension}

        \newcounter{dfsDevelLangsCCount}
        \newcounter{dfsDevelLangsJavaCount}
        \newcounter{dfsDevelLangsOtherCount}
        \newcounter{dfsDeployOnPremisesCount}
        \newcounter{dfsDeployIaasPaasCount}
        \newcounter{dfsDeploySaasCount}
        \newcounter{dfsIsSupportedCount}
        \newcounter{dfsIsOpenSourceCount}

        \renewcommand{\dfsPrint}{
                \StrSubstitute{ \dfsDevelLangsVal,@@}{ C,}{}[\dfsTmp]
                \StrSubstitute{\dfsTmp}{ Java,}{}[\dfsTmp]
                \StrSubstitute{\dfsTmp}{,@@}{}[\dfsTmp]
                \StrSubstitute{\dfsTmp}{@@}{}[\dfsTmp]
                \xdef\dfsDevelLangsOtherVal{\dfsTmp}
                \StrSubstitute{\dfsTmp}{ }{}[\dfsTmp]
                \IfStrEq{\dfsTmp}{}{\xdef\dfsDevelLangsOtherVal{}}{}
                %
                \ifdefempty{\dfsNameVal}{}{\ifbool{dfsIsAcademicVal}{\textit{\dfsNameVal}}{\dfsNameVal}} &
                \IfSubStr{ \dfsDevelLangsVal,}{ C,}{\cmark\addtocounter{dfsDevelLangsCCount}{1}}{--} &
                \IfSubStr{ \dfsDevelLangsVal,}{ Java,}{\cmark\addtocounter{dfsDevelLangsJavaCount}{1}}{--} &
                \ifdefempty{\dfsDevelLangsOtherVal}{--}{\dfsDevelLangsOtherVal\addtocounter{dfsDevelLangsOtherCount}{1}} & 
                \ifdefempty{\dfsDeployOnPremisesVal}{--}{%
                	\parbox{.6cm}{
                        \cmark\addtocounter{dfsDeployOnPremisesCount}{1}\hspace{-.05cm}%
                        \IfSubStr{ \dfsDeployOnPremisesVal,}{ Native,}{%
                            $_\textit{n}$%
                        }{}%
                        \IfSubStr{ \dfsDeployOnPremisesVal,}{ Containerized,}{%
                            $_\textit{c}$%
                        }{}
                        }\hspace{-.4cm}\,
                } &
                \ifdefempty{\dfsDeployIaasPaasVal}{--}{%
                	\parbox{.6cm}{
                        \cmark\addtocounter{dfsDeployIaasPaasCount}{1}\hspace{-.05cm}%
                        \IfSubStr{ \dfsDeployIaasPaasVal,}{ AWS,}{%
                            $_\textit{a}$%
                        }{}%
                        \IfSubStr{ \dfsDeployIaasPaasVal,}{ Azure,}{%
                            $_\textit{m}$%
                        }{}%
                        \IfSubStr{ \dfsDeployIaasPaasVal,}{ GCP,}{%
                            $_\textit{g}$%
                        }{}%
                    }\hspace{-.4cm}\,
                } &
                \ifdefempty{\dfsDeploySaasVal}{--}{%
                	\parbox{.6cm}{
                        \cmark\addtocounter{dfsDeploySaasCount}{1}\hspace{-.05cm}%
                        \IfSubStr{ \dfsDeploySaasVal,}{ AWS,}{%
                            $_\textit{a}$%
                        }{}%
                        \IfSubStr{ \dfsDeploySaasVal,}{ Azure,}{%
                            $_\textit{m}$%
                        }{}%
                        \IfSubStr{ \dfsDeploySaasVal,}{ GCP,}{%
                            $_\textit{g}$%
                        }{}%
                    }\hspace{-.4cm}\,
                } &
                \ifbool{dfsIsSupportedVal}{\cmark\addtocounter{dfsIsSupportedCount}{1}}{--} &
                \ifbool{dfsIsOpenSourceVal}{\cmark\addtocounter{dfsIsOpenSourceCount}{1}}{--} &
                \ifdefempty{\dfsYearFirstVal}{--}{\dfsYearFirstVal} &
                \ifdefempty{\dfsVersionFirstVal}{--}{\dfsVersionFirstVal} &
                \ifdefempty{\dfsYearLastVal}{--}{\dfsYearLastVal} &
                \ifdefempty{\dfsVersionLastVal}{--}{\dfsVersionLastVal} \\
        }

        \begin{mytabular}{|L{2.85cm}||C{.8cm}C{.8cm}C{.975cm}||C{.925cm}C{.8cm}C{.8cm}||C{.8cm}C{0.8cm}||>{~~}L{.7cm}L{1.05cm}L{0.6cm}L{1.05cm}|}
                \hhline{|-||---||---||--||----|}
                \tabhead
                \multirow{2}{=}{System} &
                \multicolumn{3}{M||}{Main development language} &
                \multicolumn{3}{M||}{Deployment} &
                \multirow{2}{=}{\centering Comm. support} &
                \multirow{2}{=}{\centering Open source} &
                \multicolumn{4}{M|}{Release} \\
                \hhline{|~||---||---||~~||----|}
                \tabsubhead
                & C/C++ & Java & Others &
                On-prem & IaaS/PaaS & SaaS & & &
                F.\,Year & F.\,Version & L.\,Year & L.\,Version \\
                \hhline{:=::===::===::==::====:}
                \tabbody
                \ExpandableInput{Tab/systems.tex}
                \hhline{:=::===::===::==::====:}
                \tabfoot
                Number &
                \arabic{dfsDevelLangsCCount} &
                \arabic{dfsDevelLangsJavaCount} &
                \arabic{dfsDevelLangsOtherCount} &
                \arabic{dfsDeployOnPremisesCount} &
                \arabic{dfsDeployIaasPaasCount} &
                \arabic{dfsDeploySaasCount} &
                \arabic{dfsIsSupportedCount} &
                \arabic{dfsIsOpenSourceCount} &
                -- & -- & -- & -- \\
                \hhline{|-||---||---||--||----|}
        \end{mytabular}

\end{table*}

\endinput

%%% Local Variables:
%%% mode: latex
%%% TeX-master: "../main"
%%% End:



\newpage
The following Table~\ref{Tab:DataSourcesSupported} lists the specific sources supported by each investigated data federation system, obtained from available systems' documentation and publications. Sources are classified on a \emph{local}, \emph{per-system} basis, along the source types defined in Section 6 of the original paper, with additional source information\,---\,such as the specific kind(s) of relational, graph-based or aggregate-oriented system\,---\,reported next to the source name via subscript letters (see table caption for legend).
We remark the following:
\begin{itemize}
	\item
	Some sources correspond to data access interfaces that can be configured to connect additional systems beyond the ones explicitly listed in the table.
	In particular, companies such as CData\footnote{\url{https://www.cdata.com/drivers/}} and Progress\footnote{\url{https://www.progress.com/connectors}} commercialize \emph{connectors} for the relational SQL-based JDBC, ODBC, ADO.NET and OLE DB interfaces that can be used to access a myriad of heterogeneous data sources, possibly different from the ones listed in Table~\ref{Tab:DataSourcesSupported} (\eg GraphQL sources via specific connectors\footnote{\url{https://www.cdata.com/drivers/graphql/}}) and possibly using a different data model that is transparently adapted to the relational one by the connector (\eg, via flattening of nested data).
	In Table~\ref{Tab:DataSourcesSupported}, besides the supported data access interfaces, we explicitly list only the sources that are directly and natively supported by a system without relying on such third party connectors~/ adapters.
	\item
	Structured files are distinguished from other source types with the same data model (\eg relational sources for CSV files, aggregate-oriented\,---\,specifically, document-based\,---\,for JSON files) by virtue of direct access to raw file contents by the data federation system.
	In some cases, however, access to stored structured files may require metadata services external to the filesystem (\eg Hive Metadata Store) for locating and interpreting file contents, or may leverage processing services (\eg from Hadoop) co-located with the nodes storing the file in a distributed filesystem (\eg HDFS), for instance to \emph{push down} data access operations and computations (\eg filtering, sorting) close to where raw file data reside, this way reducing communication costs.
	\item
	Some of the data federation systems investigated in this survey are also listed as supported sources (marked with $\ast$ subscript) of other systems in Table~\ref{Tab:DataSourcesSupported}, reflecting the fact that the virtual data sources obtained through data federation can be used themselves in downstream federations.
	As a limit case (\eg AllegroGraph), a system may list only itself as a supported data source, which occurs when the system offers both storage and data federation capabilities, and the latter are restricted to instances of the same system.
	\item
	Test sources (\eg emulating \texttt{/dev/null}) and system-specific connectors used to access configuration, performance or log data of the system itself are omitted in Table~\ref{Tab:DataSourcesSupported}, for simplicity.
\end{itemize}

% !TeX spellcheck = en_US

%-Standards of types of data sources

%\clearpage
%\onecolumn

{
	\renewcommand{\dfsPrint}{
		\xdef\dfsSrcPrefixDefault{, }%
		\gdef\dfsSrcPrintCell{\xdef\dfsSrcPrefixCurrent{}\ignorespaces\dfsSrcListVal}%
		\ifdefempty{\dfsNameVal}{}{\ifbool{dfsIsAcademicVal}{\textit{\dfsNameVal}}{\dfsNameVal}} &
		\xdef\dfsSrcTypeFilter{r}\dfsSrcPrintCell &
		\xdef\dfsSrcTypeFilter{g}\dfsSrcPrintCell &
		\xdef\dfsSrcTypeFilter{a}\dfsSrcPrintCell &
		\xdef\dfsSrcTypeFilter{f}\dfsSrcPrintCell &
		\xdef\dfsSrcTypeFilter{w}\dfsSrcPrintCell &
		\xdef\dfsSrcTypeFilter{o}\dfsSrcPrintCell
		\xdef\dfsSrcTypeFilter{}\xdef\dfsSrcPrefixDefault{} \\
	}

	\captionsetup[longtable]{labelfont={normal,footnotesize},labelsep=newline,textfont={normal,footnotesize},justification=centerfirst}
	\begin{mylongtable}{|L{1.4cm}||L{4.7cm}L{1.2cm}L{2.2cm}L{1.4cm}L{1.3cm}||L{2.2cm}|}
		\caption{%
			\parbox{\linewidth}{
			\vspace{.15cm}
			Supported data sources.
			Academic systems in \textit{italics}.
			Additional source information in subscript position: 
			\emph{$\ast$} = investigated system;
			\emph{a} = specialized web API;
			\emph{r} = RDF triple store;
			\emph{g} = property graph store;
			\emph{k} = key-value store;
			\emph{w} = wide-column store;
			\emph{d} = document store;
			\emph{s} = search engine;
			\emph{h} = hardware~+ software appliance;
			\emph{m} = MDX (MultiDimensional eXpressions) support.
			SPARQLp denotes the SPARQL protocol
			}
		}
	    \label{Tab:DataSourcesSupported}
	    \\
		\hhline{|-||-----||-|}
		\tabhead
		System & \srcTypeRelational & \srcTypeGraph & \srcTypeAggregate & \srcTypeFile & \srcTypeService & \srcTypeOther \\
		\hhline{:=::=====::=:}
		\endfirsthead
		\hhline{|-||-----||-|}
		\tabhead
		System & \srcTypeRelational & \srcTypeGraph & \srcTypeAggregate & \srcTypeFile & \srcTypeService & \srcTypeOther \\
		\hhline{:=::=====::=:}
		\endhead
		\hhline{|-||-----||-|}
		\endfoot
		\tabbody
		\ExpandableInput{Tab/systems.tex}
	\end{mylongtable}
}

\endinput

%%% Local Variables:
%%% mode: latex
%%% TeX-master: "../main"
%%% End:




\bibliographystyle{ios1}
\bibliography{DVSurvey}


\end{document}

\endinput


%%% Local Variables:
%%% mode: latex
%%% TeX-master: t
%%% TeX-source-correlate-mode: t
%%% End:
